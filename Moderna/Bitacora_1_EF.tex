\documentclass[12pt]{exam}
\usepackage{amsthm}
\usepackage{libertine}
\usepackage[utf8]{inputenc}
\usepackage[margin=1in]{geometry}
\usepackage{amsmath,amssymb}
\usepackage{multicol}
\usepackage[shortlabels]{enumitem}
\usepackage{siunitx}
\usepackage{cancel}
\usepackage{caption}
\usepackage{graphicx}
\usepackage{pgfplots}
\usepackage{listings}
\usepackage{tikz}


\pgfplotsset{width=10cm,compat=1.9}
\usepgfplotslibrary{external}
\tikzexternalize

\newcommand{\class}{Laboratorio Moderna} % This is the name of the course 
\newcommand{\examnum}{Laboratorio $0$} % This is the name of the assignment
\newcommand{\examdate}{\today} % This is the due date
\newcommand{\timelimit}{}
\newenvironment{Figura}
  {\par\medskip\noindent\minipage{\linewidth}}
  {\endminipage\par\medskip}




\begin{document}
\pagestyle{plain}
\thispagestyle{empty}

\noindent
\begin{tabular*}{\textwidth}{l @{\extracolsep{\fill}} r @{\extracolsep{6pt}} l}
\textbf{\class} & \textbf{Name:} & Sergio Montoya Ramírez\\ %Your name here instead, obviously 
\textbf{\examnum} &&\\
\textbf{\examdate} &&\\
\end{tabular*}\\
\rule[2ex]{\textwidth}{2pt}
% ---


\begin{multicols}{2}
\section{Introducción}
En 1921 el premio Nobel de física en el año 1921 fue para Albert Einstein por sus aportaciones a la física por la ley del efecto fotoelectrico.
\section{Objetivos}
\begin{itemize}
\item Estudiar los electrones emitidos por una placa de metal iluminada.
\item 
\end{itemize}
\section{Analisis Cualitativo}
\begin{enumerate}
\item Pregunta 1
\item Pregunta 2
\item Pregunta 3
\item Pregunta 4
\item Pregunta 5
\item Pregunta 6
\end{enumerate}
\section{Analisis Cuantitativo}
\begin{Figura}
    \centering
    \includegraphics[width=0.9\textwidth]{../Al_toque_perro.png}
    \captionof{figure}{Al toque perro}
    \label{fig}
\end{Figura}
\section{Conclusión}
Copiar lo que envie David
\end{multicols}

\end{document}
